\newpage
\begin{Large}
\textbf{Quá trình thực hiện công việc}
\end{Large}

\vspace{2cm}

\setlength\tabcolsep{10pt}
\begin{center}
\begin{longtable}{|c|c|p{8cm}|}
\hline
\textbf{STT}		& \textbf{Tuần}					& \textbf{Công việc thực hiện}	\\
\hline
1					& 18/8/2014 - 24/8/2014			& Nghiên cứu ngôn ngữ ASM, cách sử dụng công cụ Olly.\\
\hline
2					& 25/8/2014 - 31/8/2014			&  Nghiên cứu PEHeader của tập tin, cách sử dụng WinHex để đọc nội dung của tập tin và so sánh số lượng node và edge của công cụ IDA Pro.\\
\hline
3					& 1/9/2014 - 7/9/2014			& Nghiên cứu về mô hình memory, register, stack trên Win32. Đồng thời đọc và hiểu về CFG của hệ thống BE-PUM.\\
\hline
4					& 8/9/2014 - 14/9/2014			& Nghiên cứu về địa chỉ RVA, VA trong memory, hiểu về memory trong OllyDBG. Đọc mã nguồn và phân tích virus Aztec.\\
\hline
5					& 15/9/2014 - 21/9/2014			& Viết báo cáo về virus Aztec.\\
\hline
6					& 22/9/2014 - 28/9/2014			& Trình bày virus Aztec với giáo viên hướng dẫn. Tìm hiểu về Windows API và cách hiện thực trên hệ thống BE-PUM.\\
\hline
7					& 29/9/2014 - 5/10/2014			&  Tìm hiểu API trong thực tế với ngôn ngữ ASM.\\
\hline
8					& 6/10/2014 - 12/10/2014		& Đọc hiểu về kỹ thuật Self-modifying Code và cách xử lý kỹ thuật này trên hệ thống BE-PUM. Đọc và phân tích virus Sasser.\\	
\hline
9					& 13/10/2014 - 19/10/2014		& Đọc hiểu về kỹ thuật Strutured Exception Handling và cách xử lý kỹ thuật này trên hệ thống BE-PUM. Đọc và tiếp tục phân tích virus Sasser.\\	
\hline
10					& 20/10/2014 - 26/10/2014		& Đọc và tiếp tục phân tích virus Sasser. Cải tiến API trên hệ thống BE-PUM.\\	
\hline
11					& 27/10/2014 - 2/11/2014		& Đọc và tiếp tục phân tích virus Sasser. Đọc hiểu về hệ thống BE-PUM. Tìm hiểu về Capstone Engine.\\	
\hline
12					& 3/11/2014 - 9/11/2014			& Thu thập các kỹ thuật của virus Sasser với các kỹ thuật: SMC, SEH, Decryption, Import Reconstruction.\\	
\hline
13					& 10/11/2014 - 16/11/2014		& Đọc và tìm hiểu virus Cabanas, so sánh giữa Cabanas và Sasser.\\	
\hline
14					& 17/11/2014 - 23/11/2014		& Đọc và tìm hiểu virus Cabanas. Tìm hiểu công cụ packer PIN và TELOCK và các cơ chế đóng gói.\\	
\hline
15					& 24/11/2014 - 30/11/2014		& Chuẩn bị bài trình bày về nội dung đã nghiên cứu với GS. Mizuhito Ogawa bao gồm: Sasser, Cabasnas, BE-PUM, Windows API, Capstone, PIN.\\	
\hline
16					& 1/12/2014 - 7/12/2014			& Tiếp tục chuẩn bị bài trình bày. Tìm hiểu về kỹ thuật lây nhiễm Buffer Overflow của malware trên LSASS service của Windows.\\	
\hline
17					& 8/12/2014 - 14/12/2014		& Báo cáo với GS.Mizuhito Ogawa về nội dung đã nghiên cứu.\\	
\hline
18					& 15/12/2014 - 21/12/2014		& Đọc và tìm hiểu lý thuyết về Automata. Xây dựng bộ testcase cho các câu lệnh hiện thực trên hệ thống BE-PUM.\\	
\hline
19					& 22/12/2014 - 28/12/2014		& Đọc và tìm hiểu lý thuyết về Automata. Sửa lỗi các câu lệnh hiện thực trên hệ thống BE-PUM.\\	
\hline
20					& 29/12/2014 - 4/1/2015			& Phân tích Sasser và Cabanas trực tiếp trên hệ thống BE-PUM và hỗ trợ hệ thống BE-PUM với các kỹ thuật của virus.\\	
\hline
21					& 5/1/2015 - 11/1/2015			& Xây dựng công cụ để so sánh tự động công cụ IDA và hệ thống BE-PUM. Chuẩn bị sang JAIST vào ngày 12/01/2015.\\	
\hline
22					& 12/1/2015 - 18/1/2015			& Trình bày về các công việc đã nghiên cứu: Nested SEH và virus Sandman. Định hướng sẽ nghiên cứu tại JAIST.\\	
\hline
23					& 19/1/2015 - 25/1/2015			& Thực hành với Automata. Tiếp tục xây dựng công cụ tự động để so sánh IDA và BE-PUM.\\	
\hline
24					& 26/1/2015 - 1/2/2015			& Thực hành với Automata và Regular Expressions.\\
\hline
25					& 2/2/2015 - 8/2/2015			& Đọc và tìm hiểu về Model Checking, CTL, LTL. Thực hành trên công cụ NuSMV.\\
\hline
26					& 2/3/2015 - 8/3/2015			& Đọc và tìm hiểu về Pushdown System, chuẩn bị bài trình bày sau khi về từ JAIST. Thí nghiệm model checking trên NuSMV với kỹ thuật SEH của malware.\\
\hline
27					& 9/3/2015 - 15/3/2015			& Chạy thí nghiệm với tập malware sử dụng kỹ thuật SEH và không sử dụng kỹ thuật SEH. Tìm hiểu về packer và TELOCK packer.\\
\hline
28					& 16/3/2015 - 22/3/2015			& Báo cáo về kỹ thuật của packer TELOCK.\\
\hline
29					& 23/3/2015 - 29/3/2015			& Tổng kết các mẫu kỹ thuật SEH. Tìm hiểu về các kỹ thuật chung của packer, biểu diễn dưới dạng CTL và phân tích PECOMPACT packer.\\
\hline
30					& 30/3/2015 - 5/4/2015			& Phân tích PECOMPACT trên hệ thống BE-PUM. Phân tích các kỹ thuật của packer Themida và YODA. Hiện thực lớp Segment register cho công cụ model checking kết hợp hệ thống BE-PUM.\\
\hline
31					& 6/4/2015 - 12/4/2015			& Tiếp tục phân tích packer. Thí nghiệm công cụ model checking với các kỹ thuật của packer. Xây dựng công cụ hỗ trợ debug cho BE-PUM kết hợp OllyDBG.\\
\hline
32					& 13/4/2015 - 19/4/2015			& Tiếp tục phân tích packer Themida và hỗ trợ các kỹ thuật của Themida trên hệ thống BE-PUM.\\
\hline
33					& 20/4/2015 - 26/4/2015			& Phân tích paker PETITE. Hiện thực EXCEPTION\_RECORD và CONTEXT\_RECORD cho hệ thống BE-PUM để hỗ trợ phân tích packer PETITE.\\
\hline
34					& 4/5/2015 - 10/5/2015			& Phân tích paker ASPack, Fastpack. Hiện thực PEB và TIB cho hệ thống BE-PUM. Bổ sung mẫu hành vi cho kỹ thuật SEH.\\
\hline
35					& 11/5/2015 - 17/5/2015			& Phân tích các kỹ thuật của packer Fastpack.\\
\hline
36					& 18/5/2015 - 24/5/2015			& Tiếp tục hỗ trơ phân tích Fastpack. Sửa lỗi cho phần hiện thực TIB và PEB.\\
\hline
37					& 25/5/2015 - 31/5/2015			& Hiện thực các thanh ghi FS, CS, SS, DS, ES hỗ trợ hệ thống BE-PUM xử lý các packer.\\
\hline
38					& 1/6/2015 - 7/6/2015			& Phân tích các packer NPack, UPX, FSG. Tìm hiểu về các kỹ thuật Anti-reversing được sử dụng trong packer. Mô tả kỹ thuật này với CTL. Viết báo cáo cho giai đoạn thực tập tốt nghiệp.\\
\hline
39					& 8/6/2015 - 14/6/2015			& Chạy thí nghiệm model checking trên các packer. Hoàn tất báo cáo và chuẩn bị trình bày trước giáo viên phản biện.\\
\hline
40					& 15/6/2015 - 21/6/2015			& Phân tích Themida packer. Báo cáo thực tập tốt nghiệp với giáo viên phản biện.\\
\hline
41					& 22/6/2015 - 28/6/2015			& Tiếp tục phân tích các packer khác. Chạy thí nghiệm packer trên hệ thống BE-PUM, hỗ trợ các kỹ thuật của packer.\\
\hline
42					& 29/6/2015 - 5/7/2015			& Tiếp tục phân tích packer và thu thập các kỹ thuật của packer biểu diễn dưới dạng CTL.\\
\hline
43					& 6/7/2015 - 12/7/2015			& Viết báo cáo về các kỹ thuật anti-reversing của packer.\\
\hline
44					& 13/7/2015 - 19/7/2015			& Chuẩn bị trình bày với GS. Mizuhito Ogawa về các kỹ thuật của packer.\\
\hline
45					& 20/7/2015 - 26/7/2015			& Biểu diễn kỹ thuật anti-reversing với CTL. Hỗ trợ nhận dạng packer trên model checking và hệ thống BE-PUM qua On-the-fly.\\
\hline
46					& 27/7/2015 - 2/8/2015			& Bổ sung nhận dạng packer thông qua chữ ký cho hệ thống BE-PUM.\\
\hline
47					& 3/8/2015 - 9/8/2015			& Thí nghiệm nhận dạng packer thông qua chữ ký và model checking.\\
\hline
48					& 10/8/2015 - 16/8/2015			& Thí nghiệm tập virus từ LORIA. Sửa lỗi phát sinh cho quá trình nhận dạng packer On-the-fly.\\
\hline
49					& 17/8/2015 - 23/8/2015			& Báo cáo kết qủa thí nghiệm. Tiến hành phân tích thêm các packer khác.\\
\hline
50					& 24/8/2015 - 30/8/2015			& Hỗ trợ thêm nhận dạng packer thông qua On-the-fly. So sánh hệ thống BE-PUM và các công cụ METASM và HOPPER.\\
\hline
51					& 31/8/2015 - 6/9/2015			& Tiếp tục hỗ trợ nhận dạng packer. Tiến hành thí nghiệm trên 7 packer đã phân tích.\\
\hline
52					& 7/9/2015 - 13/9/2015			& Phân loại các kỹ thuật của packer.\\
\hline
53					& 14/9/2015 - 20/9/2015			& Phân tích packer TELOCK với các kỹ thuật mới: SEH và trap flag, Hardware breakpoints và CRC checking.\\
\hline
54					& 121/9/2015 - 27/9/2015		& Báo cáo packer TELOCK với giáo viên hướng dẫn. Hỗ trợ các kỹ thuật của packer TELOCK cho hệ thống BE-PUM.\\
\hline
55					& 28/9/2015 - 4/10/2015			& Sửa lỗi nhận dạng packer cho hệ thống BE-PUM.\\
\hline
56					& 5/10/2015 - 11/10/2015		& Thu thập các chữ ký của packer dưới dạng dữ liệu JSON. Bổ sung các mẫu hành vi của packer.\\
\hline
57					& 12/10/2015 - 18/10/2015		& Nhận dạng packer thông qua frequency của các kỹ thuật. Tìm vị trí entry point thực sự của tập tin đóng gói.\\
\hline
58					& 19/10/2015 - 25/10/2015		& Hiện thực nhận dạng packer thông qua frequency kỹ thuật. Chuẩn bị trình bày cho báo cáo giữa kỳ.\\
\hline
59					& 26/10/2015 - 1/11/2015		& Chỉnh sửa báo cáo giữa kỳ. Kiểm tra quá trình nhận dạng chữ ký của packer với 2 công cụ PEID và CFF Explorer.\\
\hline
60					& 2/11/2015 - 8/11/2015			& Tiếp tục bổ sung frequency của các packer PEBUNDLE và MPRESS. Bổ sung frequency cho kỹ thuật Hardware Breakpoint.\\
\hline
61					& 9/11/2015 - 15/11/2015		& Hỗ trợ BE-PUM phân tích bổ sung các packer khác, có 27 packer đã được BE-PUM hỗ trợ.\\
\hline
62					& 16/11/2015 - 22/11/2015		& Tiếp tục thí nghiệm nhận dạng packer qua chữ ký và ngữ nghĩa, cũng như qua frequency của kỹ thuật từ tập malware LORIA.\\
\hline
63					& 23/11/2015 - 29/11/2015		& Tiếp tục thí nghiệm nhận dạng packer.\\
\hline
64					& 30/11/2015 - 6/12/2015		& Chuẩn bị bài trình bày cho luận văn tốt nghiệp, cấu trúc bài luận văn. Hoàn tất bản nháp cho bài báo SEATUC.\\
\hline
65					& 7/12/2015 - 13/12/2015		& Hoàn thiện bài trình bày và chuẩn bị báo cáo luận văn tốt nghiệp. Tiếp tục viết bài báo SEATUC.\\
\hline
66					& 13/12/2015 - 20/12/2015		& Hoàn thiện báo cáo luận văn tốt nghiệp. Tiếp tục viết bài báo SEATUC.\\
\hline
67					& 20/12/2015 - 27/12/2015		& Chuẩn bị bài trình bày với giáo viên phản biện, hoàn thiện tiếp tục luận văn.\\
\hline
68					& 28/12/2015 - 3/1/2015			& Báo cáo luận văn tốt nghiệp với giáo viên phản biện.\\
\hline
69					& 4/1/2015 - 11/1/2015			& Báo cáo trước hội đồng luận văn tốt nghiệp.\\
\hline
\end{longtable}
\end{center}

