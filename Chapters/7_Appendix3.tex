\newpage
\begin{Large}
\textbf{Trích dẫn bài báo}
\end{Large}

\vspace{2cm}

\begin{Large}
\begin{center}
\textbf{PRECISE PACKER DETECTION\\ USING MODEL CHECKING}
\end{center}
\end{Large}

\vspace{1cm}

\begin{center}
\textbf
{
Nguyen Minh Hai, Quan Thanh Tho, Do Duy Phong, Le Duc Anh\\ 
Ho Chi Minh City University of Technology, Vietnam\\
hainmmt@cse.hcmut.edu.vn, qttho@cse.hcmut.edu.vn, doduyphongbktphcm@gmail.com, tintinkool@gmail.com
}
\end{center}

\vspace{1cm}

\textbf{Extended Abstract}\\
Over the past decades, malware has been becoming a real threat. It costs more than \$10 billion in each year and the damage is still increasing. Most of the modern popular malwares are either packed or obfuscated. The main goal of these obfuscation techniques is to thwart the signature based technique of anti-virus software. It also increases the difficulty of the reverse engineering work since it often takes a very long time for unpacking or decrypting a packed file. As a counter solution, most of anti-virus software tends to detect packer signature for verifying the packed malware. However, since hacker can easily modify signature header of packed file, this solution cannot determine precisely whether a malware is packed or not. This paper proposes a model checking method for packer detection using a combination BE-PUM tool and model checker NUSMV. BE-PUM (Binary Emulator for PUshdown Model generation) is designed for generating a precise control flow graph (CFG), under presence of typical obfuscation techniques of malware, e.g., indirect jump, self-modification, overlapping instructions, and structured exception handler (SEH), which are supported in packers. Currently, BE-PUM can cover the patterns for 13 techniques mainly used in 8 packers i.e. UPX, FSG, NPACK, ASPACK, PECOMPAT, PETITE, YODA and TELOCK. Applying the CTL formula for that patterns as properties of proposed model checker tool, we can detects totally all the malwares which are packed by these packers. We have implemented our technique and applied to automatically detect packed malware. The experiment results are encouraging.

\end{document}