\setlength\parindent{0pt}
\hspace{0.5cm}Hiện nay, an toàn thông tin là vấn đề vô cùng quan trọng và đóng vai trò to lớn, những phần mềm độc hại (malicious software - malware) đã và đang dần trở thành mối đe doạ thực sự với vấn đề an toàn thông tin của mỗi quốc gia. Vấn đề ngăn chặn, cũng như khắc phục những hậu quả để lại của malware là vô cùng lớn và tiêu tốn hơn 10 tỉ đô la mỗi năm, con số này vẫn tiếp tục gia tăng do mức độ độc hai và tinh vi của malware ngày càng gia tăng. Nguy hiểm hơn, malware còn được sử dụng như một công cụ nhằm xâm nhập và đánh cắp thông tin người dùng một cách bất hợp pháp.\\  

\hspace{0.5cm}Đa phần những phần mềm chống virus nói riêng và malware nói chung được sử dụng trong công nghiệp có thể phát hiện những phần mềm độc hại này thông qua kỹ thuật chữ ký, hay nói cách khác mỗi malware sẽ có một chữ ký duy nhất và có thể được phát hiện dễ dàng. Tuy nhiên, với những malware ngày này việc sử dụng một phần mềm đóng gói (packer) nhằm khiến những phần mềm chống malware trở nên vô hại. Và hơn hết, với các malware đã được đóng gói, kỹ thuật dịch ngược vốn được xem là kỹ thuật hiệu quả để phát hiện malware cũng trở nên khó khăn và nhiều thách thức bởi bản thân packer đã sử dụng rất nhiều kỹ thuật obfuscation, cũng như kỹ thuật anti-reversing, anti-debugging ngày càng hiệu quả hơn. Do đó những phần mềm chống malware đã chọn phương thức phát hiện những malware được đóng gói thông qua chữ ký của các packer, tuy nhiên, một malware có thể sử dụng những kỹ thuật nhằm thay đổi chữ ký này và vì thế kỹ thuật này cũng sẽ rất khó khăn để xác định rằng malware đó có đang được đóng gói bởi một packer hay không.\\

\hspace{0.5cm}Xuất phát từ những khó khăn thực tế đó, đề tài luận văn tốt nghiệp này hướng tới việc xây dựng phương pháp hình thức kiểm tra mô hình (model checking) nhằm phát hiện packer bằng cách kết hợp giữa công cụ BE-PUM và công cụ kiểm tra mô hình NuSMV. Công cụ BE-PUM (Binary Emulator for PUshdown Model generation) được thiết kế để có thể xây dựng được mô hình chính xác của một chương trình mã nhị phân, BE-PUM đã có thể xử lý được những kỹ thuật obfuscation đặc trưng của malware như: indirect jump, self-modifcation code, và structured exception handler, mà những kỹ thuật này cũng đã được sử dụng trong các packer. Hiện tại, BE-PUM đã xây dựng được mẫu hành vi của 13 kỹ thuật được sử dụng chính yếu trong 27 packer mà BE-PUM đã phân tích hoàn toàn. Bằng cách biểu din những hành vi này dưới dạng công thức CTL, LTL như thuộc tích đầu vào của công cụ NuSMV, BE-PUM đã có thể phát hiện hoàn toàn được 27 packer. Cùng với việc hiện thực kỹ thuật phát hiện packer thông qua ngữ nghĩa, kỹ thuật phát hiện chữ ký cũng được xây dựng trong hệ thống BE-PUM nhằm đưa ra những so sánh cụ thể nhất về hiệu quả của từng kỹ thuật.\\

Nội dung của luận văn tốt nghiệp bao gồm các chương chính như sau:

\begin{itemize}[leftmargin=*]
\item{Mở đầu:\\
\hspace*{0.5cm}Giới thiệu sơ lược về đề tài luận văn, bố cục chung của luận văn, trình bày những mục tiêu khi chọn đề tài.\\}
\item{Chương 1: Giới thiệu\\
\hspace*{0.5cm}Chương 1 sẽ tập trung giới thiệu tổng quan về đề tài, cụ thể về hệ thống BE-PUM và cách thức xây dựng mô hình của BE-PUM, đồng thời giới thiệu về packer, phương pháp kiểm tra mô hình và và công cụ kiểm tra mô hình NuSMV; mục tiêu cụ thể hướng tới của đề tài và phạm vi hiện thực của đề tài.\\}
\item{Chương 2: Phân tích vấn đề\\
\hspace*{0.5cm}Chương 2 sẽ tập trung phân tích vấn đề, cụ thể là những vấn đề của hệ thống BE-PUM khi xử lý packer thông qua 2 ví dụ cụ thể về 2 packer được malware sử dụng trong thực tế, rút ra nhận xét tổng quát và những thách thức đặt ra cho đề tài. Ngoài ra, chương này cũng sẽ tập trung bàn luận về những thách thức này, từ đó đưa ra những giải pháp cụ thể để giải quyết những khó khăn đó.\\}
\item{Chương 3: Kiến thức nền\\
\hspace*{0.5cm}Chương 3 sẽ giới thiệu những kiến thức nền và các công nghệ được sử dụng trong đề tài. Để có thể giải quyết những thách thức như chương 2 đã nêu ra, cần nắm vững kiến thức nền tảng về hệ thống BE-PUM và lý thuyết về model checking và công cụ kiểm tra mô hình NuSMV. Ngoài ra chương này cũng tập trung giới thiệu cụ thể về 13 kỹ thuật được hỗ trợ trong packer.\\}
\item{Chương 4: Thiết kế và xây dựng\\
\hspace*{0.5cm}Chương 4 sẽ trình bày về các công việc đã thực hiện trong luận văn dựa trên những kiến thức nền được giới thiệu ở chương 2. Cụ thể, chương này sẽ tập trung giới thiệu việc kết hợp giữa hệ thống BE-PUM và công cụ kiểm tra mô hình NuSMV trong việc phát hiện packer. Để có thể so sánh giữa phương pháp phát hiện packer thông qua nhận dạng chữ ký và phương pháp kiểm tra mô hình, giải thuật phát hiện packer thông qua chữ ký cũng được hiện thực và giới thiệu cụ thể trong chương này.\\}
\item{Chương 5: Thí nghiệm, đánh giá và kết luận\\
\hspace*{0.5cm}Chương 5 sẽ chủ yếu trình bày những kết quả đã đạt được sau khi thực hiện đề tài, kết quả thí nghiệm trên tập malware, cũng như những so sánh giữa 2 phương pháp phát hiện packer.\\}
\item{Chương 6: Hướng phát triển trong tương lai\\
\hspace*{0.5cm}Chương 6 sẽ nêu ra những hạn chế được đặt ra trong quá trình thực hiện đề tài và hướng giải quyết trong tương lai để có thể khắc phục những hạn chế đó.}
\end{itemize}
