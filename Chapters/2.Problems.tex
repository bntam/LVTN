\section{BE-PUM và những khó khăn}

	\subsection{Mục tiêu hướng đến của BE-PUM}

	\subsection{Các hướng tiếp cận}

	\subsection{Mô hình luồng điều khiển}

	\subsection{Những đòi hỏi trong quá trình hiện thực}

%%%%%%%%%%%%%%%%%%%%%%%%%%%%%%%%%%%%%%%%%%%%%%%%%%%%%
%%%%%%%%%%%%%%%%%%%%%%%%%%%%%%%%%%%%%%%%%%%%%%%%%%%%%
%%%%%%%%%%%%%%%%%%%%%%%%%%%%%%%%%%%%%%%%%%%%%%%%%%%%%

\section{Các câu lệnh hợp ngữ}

%%%%%%%%%%%%%%%%%%%%%%%%%%%%%%%%%%%%%%%%%%%%%%%%%%%%%
%%%%%%%%%%%%%%%%%%%%%%%%%%%%%%%%%%%%%%%%%%%%%%%%%%%%%
%%%%%%%%%%%%%%%%%%%%%%%%%%%%%%%%%%%%%%%%%%%%%%%%%%%%%

\section{Windows API}
	\subsection{Windows API trong những phần mềm độc hại}

Để cung cấp sức mạnh và sự tiện lợi cho lập trình viên trong việc viết ứng dụng chạy trên hệ điều hành Windows, các API trong bộ Windows API mở ra nhiều cách thức nhanh chóng và mạnh mẽ cho lập trình viên trong việc tương tác với hệ thống.\\

Và vấn đề gì cũng có hai mặt của nó, sự hỗ trợ mạnh mẽ đó cũng là con đường đơn giản để các tin tặc áp dụng vào việc xây dựng nên các phương pháp tấn công, cũng như cho ra đời những phần mềm nguy hại (malware), để lại bao hậu quả xấu cho hệ thống máy vi tính trên toàn cầu.\\

Trong quá trình xây dựng BE-PUM và qua việc phân tích hàng ngàn mẫu malware chạy trên môi trường Windows được phát tán ở khắp nơi trên thế giới, hầu hết những mẫu malware trên đều áp dụng lời gọi Windows API vào cách thức tấn công của chúng. Những phương pháp tấn công phổ biến như SEH hay phương pháp chống phát hiện đều có sự tồn tại của Windows API trong đó.\\

Do đó, việc xây dựng một bộ công cụ xử lý những thông tin trả về từ Windows API là rất cần thiết cho việc phát triển hệ thống BE-PUM, một hệ thống tập trung vào phân tích mã nhị phân của malware.\\


%%%%%%%%%%%%%%%%%%%%%%%%%%%%%%%%%%%%%%%%%%%%%%%%%%%%%


	\subsection{BE-PUM và Windows API}

Mã nguồn của những API trong bộ Windows API được tập đoàn Microsoft giữ kín và không hề công bố. Chỉ có những đặc tả và hướng dẫn sử dụng được Microsoft phổ biến rộng rãi cho lập trình viên. Nghĩa rằng ta chỉ có thể biết được đầu vào của lời gọi và mong muốn đầu ra sẽ như ý, chứ không thể nắm rõ lô-gíc xử lý bên trong của chúng. Điều đó khiến cho việc xử lý đúng đắn một cách tổng quát đối với mọi đầu vào của mỗi API bằng cách viết lại bộ mã xử lý tương ứng của chúng vào trong BE-PUM dường như trở nên không thể.\\

Hướng tiếp cận hiện tại là tiến hành lấy nội dung bộ nhớ, nội dung các đối số nằm trên stack bên trong BE-PUM và tiến hành gọi thực sự với Windows API, nhận kết quả trả về và nạp lại vào trong BE-PUM để tiếp tục tiến hành phân tích các câu lệnh tiếp theo.\\

BE-PUM là một dự án được phát triển lên từ nhân của dự án JakStab và được viết hoàn toàn trên ngôn ngữ lập trình Java. Với Windows API thì lại là một câu chuyện hoàn toàn khác, Windows API được phát triển chủ yếu tập trung vào ngôn ngữ lập trình C kèm với các mô tả và cấu trúc dữ liệu được viết trên đó. Thêm lần nữa, việc hiện thực ý tưởng gọi để lấy kết quả Windows API từ trong lòng BE-PUM gặp nhiều khó khăn. Đặc biệt là việc ánh xạ các dữ liệu kiểu cấu trúc giữa hai thành phần trên cũng là một trăn trở.\\

Vì những lý do trên, cần tìm hiểu một cách thức giải quyết vấn đề nhanh chóng và đơn giản hơn bằng một bộ công cụ nào đó để xử lý rào cản ngôn ngữ giữa Java và C. Thêm vào đó, bộ công cụ này cũng cần có tính linh hoạt và mềm dẻo để cho việc phát triển về sau được dễ dàng.\\


%%%%%%%%%%%%%%%%%%%%%%%%%%%%%%%%%%%%%%%%%%%%%%%%%%%%%


	\subsection{Truy xuất Windows API bên trong BE-PUM thông qua JNA}

Vấn đề trên được giải quyết thông qua bộ thư viện Java Native Access (JNA).\\

Java Native Access là một thư viện được cộng đồng phát triển, nhằm giúp cho các chương trình được viết bằng ngôn ngữ lập trình Java dễ dàng truy cập vào các thư viện native shared mà không cần thông qua Java Native Interface. Thiết kế của JNA cũng cung cấp khả năng này mà không cần bỏ ra nhiều công sức.\\

Với khả năng ánh xạ dễ dàng giao diện lập trình giữa hai ngôn ngữ Java và C; bao gồm ánh xạ tên hàm, kiểu dữ liệu trả về, kiểu dữ liệu của các thông số đầu vào; từ những kiểu dữ liệu cơ bản đến những kiểu dữ liệu cấu trúc và kể cả con trỏ; đó là những ưu điểm để lựa chọn JNA áp dụng vào trong việc giải quyết yêu cầu của đề tài nêu trên.