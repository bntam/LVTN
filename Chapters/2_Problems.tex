% ==========================================
% PROBLEMS AND CHALLENGES
% ==========================================

\newpage
\chapter{PHÂN TÍCH VẤN ĐỀ}

\begin{concept}[15cm]
\textit{Nội dung chương này sẽ tập trung phân tích vấn đề được đặt ra trong đề tài, cụ thể là những vấn đề của hệ thống BE-PUM khi xử lý packer thông qua 2 ví dụ cụ thể về 2 packer được malware sử dụng trong thực tế, rút ra nhận xét tổng quát và những thách thức đặt ra cho đề tài. Ngoài ra, chương này cũng sẽ tập trung bàn luận về những thách thức này, từ đó đưa ra những giải pháp cụ thể để giải quyết những khó khăn đó.}
\end{concept}

\section{Ví dụ thực tế}

\subsection{Ví dụ 1}

\hspace{0.5cm}Trong thực tế, một malware để tránh bị phát hiện bởi một chương trình chống malware, có thể sử dụng packer để ẩn thân hay nói cách khác che dấu điểm nhập thực sự của mình. Cụ thể malware sẽ sử dụng packer để đóng gói đoạn code thực thi thực sự của mình, và tự mở gói bằng kỹ thuật self-unpacking. Ví dụ 1 sẽ trình bày cụ thể về packer FSG được malware sử dụng trong thực tế có sử dụng kỹ thuật packing và self-unpacking:

% COPY CODE IN HERE -> CFG

\subsection{Ví dụ 2}

\hspace{0.5cm}Đối với những packer hiện đại ngày nay, không chỉ nâng cấp độ phức tạp của giải thuật đóng gói bằng việc sử dụng rất nhiều câu lệnh tính toán phức tạp, những packer này còn áp dụng rất nhiều kỹ thuật anti-reversing, anti-debugging nhằm khiến cho quá trình dịch ngược hay gỡ lỗi trở nên khó khăn rất nhiều. Ví dụ 2 sẽ trình bày cụ thể về packer TELOCK được malware sử dụng trong thực tế có sử dụng kỹ thuật Hardware Breakpoint nhằm che dấu quá trình mở gói:

% COPY CODE IN HERE

\subsection{Nhận xét và những thách thức đặt ra}

\hspace{0.5cm}Hai ví dụ được đua ra ở trên là hai kỹ thuật nổi bật trong 2 nhóm kỹ thuật gồm: nhóm kỹ thuật obfuscation, nhóm kỹ thuật anti-reversing mà những kỹ thuật này được sử dụng rất phổ biến và là đặc trưng trong hầu hết các packer, những kỹ thuật này sẽ cản trở quá trình phân tích dịch ngược, gỡ lỗi nhằm che dấu điểm nhập chính thực sự của chương trình. Đối với những chương trình chỉ hỗ trợ quá tình phân tích tĩnh tập tin thực thi như: Jackstab, Capstone, Unicorn hay IDAPro vốn không xử lý được kỹ thuật self-modification code, là kỹ thuật lõi trong kỹ thuật unpacking của packer, hay không hỗ trợ việc xử lý, truy cập vào vùng memory nói chung và stack nói riêng, sẽ không thể hoặc hoàn toàn không có khả năng để xử lý các kỹ thuật trên. Do đó việc sử dụng một chương trình cho phép phân tích động, cụ thể là hệ thống BE-PUM là cần thiết và là giải pháp chung để phân tích một packer. Tuy nhiên, hệ thống BE-PUM còn một số hạn chế về xử lý các kỹ thuật anti-reversing, anti-debugging. Nhìn chung, những thách thức lớn nhất được đặt ra trong đề tài khi áp dụng hệ thống BE-PUM bao gồm:

\begin{itemize}
\item{Packer sử dụng rất nhiều kỹ thuật đa dạng nhằm nhận dạng nếu chương trình thực thi đang được phân tích trong môi trường gỡ lỗi bằng cách sử dụng các kỹ thuật thuộc nhóm kỹ thuật anti-reversing, anti-debugging. Vậy làm thế nào để có thể vượt qua được những kỹ thuật này.\\}
\item{Ngoài ra, packer cũng sử dụng các kỹ thuật thuộc nhóm kỹ thuật obfuscation với rất nhiều những biến thể khác nhau. Vậy làm cách nào để có thể hỗ trợ tốt hơn cho hệ thống BE-PUM để xử lý tốt hơn và hiệu quả hơn các kỹ thuật này.\\}
\item{Một thách thức khác được đặt ra là làm thế nào để có thể xác định được một tập tin thực thi có đang được đóng gói bởi một packer nào hay không, nhận dạng một packer thông qua chữ ký và thông qua ngữ nghĩa, phương pháp nào đưa ra được kết quả chính xác và tổng quát hơn.}
\end{itemize}

\section{Phân tích}

\subsection {Phân tích Packer trên hệ thống BE-PUM}

\subsection {Nhận dạng Packer}

\section{Giải pháp đề nghị}

\subsection {Thu thập và xử lý các kỹ thuật Packer}

\subsection {Nhận dạng Packer thông qua chữ ký}

\subsection {Nhận dạng Packer thông qua ngữ nghĩa}

