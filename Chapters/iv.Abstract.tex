Công nghệ thông tin đang ngày càng thể hiện được sự vượt trội trong hỗ trợ công việc hằng ngày của đời sống con người. Máy vi tính ngày một phổ biến và được áp dụng rộng rãi hơn trong mọi lĩnh vực của đời sống. Các phần mềm giúp con người xử lý công việc nhanh hơn, chính xác hơn, tiện lợi hơn. Nhưng bên cạnh các phần mềm mang lại đầy lợi ích đó, có một số lượng lớn các phần mềm đi ngược lại với mục đích ban đầu mà con người đề ra, gây ảnh hưởng xấu hoặc để trục lợi cá nhân cho tác giả viết ra chúng. Đó là các phần mềm độc hại hay còn được gọi là malware.\\

Các malware được sinh ra ngày một nhiều do sự cần thiết và phổ biến của máy vi tính, và công cuộc chống lại chúng đòi hỏi tiêu tốn rất nhiều sức lực và tiền của. Nó đã trở thành một mảng công nghệ lớn của xã hội hiện nay, với mục đích ngăn ngừa, phòng chống và tiêu diệt những tác nhân xấu có thể ảnh hưởng đến bạn thông qua công nghệ thông tin.\\

Nắm bắt được tình hình và xu hướng đó, công cụ BE-PUM đã được ra đời với mục đích là phân tích một tập tin nhị phân, mang lại những góc nhìn chi tiết bên trong chúng và cuối cùng là đánh giá xem đây có phải là một malware hay không. Đề tài luận văn này được đề ra để hợp sức cùng hỗ trợ và xây dựng thêm cho công cụ BE-PUM đó. Và để trình bày chi tiết về mọi thứ liên quan, nội dung của luận văn này sẽ được trình bày chi tiết với những mục như sau:

\newpage
\begin{description}
  	\item[Chương 1] \hfill \\
	Giới thiệu tổng quan về BE-PUM, điểm mạnh của công vụ này và so sánh với công cụ khác, những đòi hỏi trong quá trình hiện thực công cụ và yếu tố quyết  định để cho ra đề tài này; dẫn nhập về hợp ngữ assembly, dẫn nhập về Windows API, các thành phần sẽ được áp dụng để phát triển cho BE-PUM; nêu ra mục tiêu của đề tài và giới hạn trong phạm vi luận văn tốt nghiệp. \\
 	\item[Chương 2] \hfill \\
	Đem đến những cái nhìn về những vấn đề đã và đang được lưu tâm khi thực hiện đề tài này; sự phổ biến của Windows API trong những phần mềm độc hại để thấy sự cần thiết của việc xây dựng một bộ xử lý Windows API cho BE-PUM; những khó khăn khi thực hiện điều đó và giải pháp cho vấn đề. Đông thời phân tích vấn đề đặt ra là hiện thực các câu lệnh trong hợp ngữ assembly, giải thích tại sao sử dụng ngôn ngữ assembly để tiến hành phân tích chương trình.\\
	\item[Chương 3] \hfill \\
	Trình bày những kiến thức cần thiết cho quá trình thực hiện đề tài; từ những kiến thức phải nắm được về hệ thống BE-PUM do đây là một đề tài làm việc dựa trên đó; và mỗi khi làm việc với một thư viện bất kỳ, đòi hỏi ta phải tìm hiểu cách thức làm việc với thư viện đó và cả những kiến thức cần thiết do bộ thư viện ấy yêu cầu. Đồng thời cũng trình bày các kiến thức cơ bản về hợp ngữ assembly, từ đó có cái nhìn tổng quan về assembly để tiến hành xây dựng chương trình BE-PUM. \\
	\item[Chương 4] \hfill \\
	Mỗi chương trình bất kỳ đều cần một thiết kế tốt để giúp cho việc xây dựng dễ dàng và quy chuẩn hơn. Mục này sẽ trình bày cách mà bộ xử lý Windows API đã được hiện thực để tương tai sau này có thể dễ dàng sửa chữa, bảo trì và bổ sung thêm vào kiến trúc đó. Để hiểu rõ cấu chương trình mô phỏng câu lệnh assembly, sơ đồ class trình bày trong chương này thể hiện được mối quan hệ giữa các class, cấu trúc của chương trình BE-PUM được phát triển dựa trên dự án JakStab, giới thiệu các class quan trọng của chương trình BE-PUM.\\
	\item[Chương 5] \hfill \\
	 Trình bày và phân tích, so sánh với các công cụ khác về kết quả mà đề tài đã đạt được sau khi tiến hành thực hiện công việc, kết quả tổng thể mà bộ xử lý đã được đóng góp để hỗ trợ cho hệ thống BE-PUM.\\
	\item[Chương 6] \hfill \\
	 Sau khi có được những kết quả ở thời điểm hiện tại của quá trình thực hiện đề tài, chương cuối này sẽ trình bày về những phương hướng cần tiếp tục phát triển trong tương lai đối với đề tài hiện tại.\\
	\item[Phụ Lục] \hfill \\
	 Liệt kê về những tài liệu và nguồn tham khảo có liên quan đến đề tài này.\\
\end{description}
