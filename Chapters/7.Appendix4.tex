\fancyhead[R]{\slshape PHỤ LỤC}
\begin{center}
	\begin{huge}
			\textbf{PHỤ LỤC 4}\\
			\textit{Bảng công việc}
	\end{huge}
\end{center}

	\begin{longtable}{|l|m{11cm}|}
		\hline
			Tuần & Nội dung công viêc \\
		\hline
		\hline
			2/3 - 8/3	& Thực hiện test case với số câu lệnh hiện có trong BE-PUM. Hiện thực các Windows API thêm cho bộ xử lý.\\
		\hline
			9/3 - 15/3	& Hiện thực các Windows API thêm cho bộ xử lý. Sửa lỗi với các Windows API trước gặp vấn đề.\\
		\hline	
			16/3 - 5/4&	Nghiên cứu tài liệu về câu lệnh xử lý số nguyên, đồng thời hỗ trợ anh Hải trong một số công việc. Tiếp tục hiện thực các Windows API thêm cho bộ xử lý, sửa lỗi với các Windows API trước gặp vấn đề. Bắt đầu hiện thực bộ quản lý môi trường tương tác vật lý cho BE-PUM (storage).\\
		\hline	
			5/4 - 12/4&	Mô phỏng câu lệnh CMOVcc. Bên cạnh đó là việc chạy thí nghiệm để rút được kết quả trong quá trình làm việc.\\
		\hline	
			13/4 - 17/5	&Mô phỏng câu lênh  BSWAP, XADD, CMPXCHG, CMPXCHG8B, CWD/CDQ, CBW, CWDE, SHRD, SHRL, RCL, RCR,  BT, BTS,... Xây dựng trang web cho dự án BE-PUM. Sửa lỗi và hiện thực thêm cho bộ Windows API; tìm hiểu về IAT (Import Address Table).\\		
		\hline	
			18/5 -31/5	&Thực hiện test với các câu lệnh đã được mô phỏng. Gặp vấn đề và sửa lỗi cho bộ quản lý môi trường tương tác vật lý cho BE-PUM. Do gặp lỗi với Windows API khi chạy một số packer, tiến hành điều tra và cập nhật sửa lỗi.\\
		\hline	
			1/6 - 14/6	& Viết và hoàn thiện báo cáo thực tập tốt nghiệp. Bên cạnh đó vẫn hỗ trợ anh Hải trong quá trình tìm tài liệu và bổ sung kiế thức để phục vụ cho quá trình luận văn tốt nghiệp.\\	
		\hline	
			3/8 - 23/8&	Thực hiện test với các câu lệnh xử lý số nguyên. Hỗ trợ tiếp các Windows API mà anh Hải giao và sửa lỗi cho các Windows API mới trước đó gặp vấn đề.\\
		\hline	
			24/8 - 13/9&	Nghiên cứu tài liệu về FPU, đồng thời hỗ trợ anh Hải trong một số công việc. Tiến hành chạy thí nghiệm trên tập malware mới từ Loria. Vấn liên tục cập nhật các Windows API còn sau khi chạy thí nghiệm.\\
		\hline	
			14/9 - 27/9	& Hiện thực số thực và gặp vấn đề trong biểu diễn số thực. Sau khi nhận thấy có dấu hiệu với hiệu năng của BE-PUM bị giảm sút nên đã tiến hành kiểm tra lại những phần thường xuyên được truy cập và tiến hành tối ưu chúng.\\
		\hline	
			28/9 - 4/10&	Giải quyết vấn đề biểu diễn số thực. Tốc độ làm việc của BE-PUM sau quá trình kiểm tra và tối ưu đã nâng lên ít nhất 2 lần. Tăng cường hỗ trợ các Windows API cho BE-PUM.\\
		\hline	
			5/10 - 11/10&	Biểu diễn số thực dạng nhị phân, lưu trữ được vào trong stack thanh ghi. Trong thời gian của chuyến thực tập ở JAIST, Nhật Bản và làm việc theo yêu cầu của thầy Ogawa, chuẩn bị một buổi thuyết trình về các công việc đã làm với BE-PUM.\\
		\hline	
			12/10 - 18/10&	Mô phỏng các câu lệnh FADD, FADDP, FIADD, FSUB, FSUBP, FSUBR, FSUBRP, FISUB, FISUBR, FMUL,... Trong tuần đã có một buổi thuyết trình mà thầy Ogawa yêu cầu. Sau đó được thầy hỗ trợ và yêu cầu thảo luận thêm về hướng phát triển kết tiếp trong công việc, bộ sinh mã tự động cho Windows API với anh Lê Vinh -- nghiên cứu sinh tại đây.\\
		\hline	
			19/10 - 25/10&	Mô phỏng các câu lệnh  FDIVRP, FIDIVR, FABS, FCHS, FSQRT, FPREM, FPREM1, FRNDINT, FXTRACT, FSIN, FCOS, FSINCOS,... Do gặp lỗi khi chạy BE-PUM trong máy ảo nên đã tìm cách khắc phục vấn đề này. Tiếp tục cùng anh Lê Vinh đưa ra hướng giải quyết cho công cụ.\\
		\hline
			26/10 - 15/11	& Mô phỏng các câu lệnh FCOM, FCOMP, FCOMPP, FUCOM, FUCOMP, FUCOMPP, FICOM,  FTST, FXAM... Thực hiện tiếp việc chạy thí nghiệm với bộ Loria. Nhận danh sách các Windows API cần hỗ trợ và hiện thực chúng cho BE-PUM. Sau quá trình thảo luận với anh Lê Vinh, thống nhất ý kiến và hiện thực sơ bộ thì có một buổi trình bày công việc với thầy Ogawa cùng mọi người về kết quả công việc.\\		
		\hline	
			16/11 - 29/11&	Tiến hành test để kiểm tra sự chính xác của các câu lệnh được mô phỏng. Kết thúc chuyến thực tập tại JAIST, Nhật Bản và vẫn tiếp tục hỗ trợ thêm một danh sách dài các Windows API. Các Windows API mới hơn đã có làm việc với bộ mã Unicode nên BE-PUM cần chỉnh sửa để làm việc được với Unicode.\\
		\hline	
			30/11 - 6/12	& Viết báo cáo luận văn. Sau buổi nghe về công việc của cô Petrucci đến từ Pháp, nhận thấy được điểm tích cực của việc mở rộng giải thuật On-The-Fly của BE-PUM sang bước kế tiếp đó là hỗ trợ chạy trên multi-thread nên đã tiến hành thực hiện công việc này.\\
		\hline	
			7/12 - 13/12	& Kiểm tra báo cáo luận văn.\\
		\hline	
			14/12 - 20/12&	Hoàn thiện báo cáo luận văn.\\
		\hline
			\caption{Bảng công việc}			
			
	\end{longtable}