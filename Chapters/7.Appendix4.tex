\fancyhead[R]{\slshape PHỤ LỤC}
\begin{center}
	\begin{huge}
			\textbf{PHỤ LỤC 1}\\
			\textit{Bảng công việc}
	\end{huge}
\end{center}

	\begin{longtable}{|l|m{13cm}|}
		\hline
			Tuần & Nội dung công viêc \\
		\hline
		\hline
			2/3 - 8/3	& Thực hiện test case với số câu lệnh hiện có trong BE-PUM\\
		\hline	
			16/3 - 5/4&	Nghiên cứu tài liệu về câu lệnh xử lý số nguyên, đồng thời hỗ trợ anh Hải trong một số công việc.\\
		\hline	
			5/4 - 12/4&	Mô phỏng câu lênh CMOVcc\\
		\hline	
			13/4 - 17/5	&Mô phỏng câu lênh  BSWAP, XADD, CMPXCHG, CMPXCHG8B, CWD/CDQ, CBW, CWDE, SHRD, SHRL, RCL, RCR,  BT, BTS, ....\\		
		\hline	
			18/5 -31/5	&Thực hiện test với các câu lệnh đã được mô phỏng\\
		\hline	
			1/6 - 14/6	&Viết  và hoàn thiện báo cáo  thực tập tốt nghiệp.\\	
		\hline	
			3/8 - 23/8&	Thực hiện test với các câu lệnh xử lý số nguyên.\\
		\hline	
			24/8 - 13/9&	Nghiên cứu tài liệu về FPU, đồng thời hỗ trợ anh Hải trong một số công việc.\\
		\hline	
			14/9 - 27/9	&Hiện thực số thực. Gặp vấn đề trong biểu diễn số thực.\\
		\hline	
			28/9 - 4/10&	Giải quyết vấn đề biểu diễn số thực.\\
		\hline	
			5/10 - 11/10&	Biểu diễn số thực dạng nhị phân, lưu trữ được vào trong stack thanh ghi.\\
		\hline	
			12/10 - 18/10&	Mô phỏng các câu lệnh FADD, FADDP, FIADD, FSUB, FSUBP, FSUBR, FSUBRP, FISUB, FISUBR, FMUL, ..\\
		\hline	
			19/10 - 25/10&	Mô phỏng các câu lệnh  FDIVRP, FIDIVR, FABS, FCHS, FSQRT, FPREM, FPREM1, FRNDINT, FXTRACT, FSIN, FCOS, FSINCOS,..\\
		\hline
			26/10 - 15/11	&Mô phỏng các câu lệnh FCOM, FCOMP, FCOMPP, FUCOM, FUCOMP, FUCOMPP, FICOM,  FTST, FXAM...\\		
		\hline	
			16/11 - 29/11&	Tiến hành test để kiểm tra sự chính xác của các câu lệnh được mô phỏng.\\
		\hline	
			30/11 - 6/12	&Viết báo cáo luận văn.\\
		\hline	
			7/12 - 13/12	&Kiểm tra báo cáo luận văn.\\
		\hline	
			14/12 - 20/12&	Hoàn thiện báo cáo luận văn.\\
		\hline
			\caption{Bảng kế hoạch nghiên cứu hiện thực câu lệnh assembly}
	\end{longtable}